Like ROC analysis, FROC analysis is also performed on 3D probability maps. However, FROC analysis is compared against biopsy results rather than the cancerous lesion contours. First non-maximum suppression (NMS) is performed in 3D on the probability maps to produce a set of candidate detections. The NMS window used was $10\ mm \times 10\ mm \times\ 10mm$. The candidates roughly reflect local maxima in the probability map. Next, a bipartite graph is formed between detection candidates and ground truth biopsy points. An edge is placed between a detection candidate and ground truth if it is within $10\ mm$. Additionally the edges are weighted with the following heuristic
\begin{equation}
w = p \times 1.5^{g - 6}
\end{equation}
where $w$ is the edge weight, $p$ is the detection probability, and $g$ is the recorded Gleason score ranging from 6-10. The special Gleason score $g = 0$ is used for benign biopsy points. The heuristic attempts to contrast detections paired with severe cancers to detections paired with less severe cancer or benign biopsies while also considering the detection probability. Finally a weighted maximum matching is determined on the bipartite graph. The matched detections then count toward detection rate. Non-matched detection candidates with at least one edge are not counted as these would be redundant detections for one of the biopsies. Non-matched detection candidates with no edge are counted as false positives. Biopsy points that are not matched are counted toward false negative rate. Similar to the final ROC curve, the FROC final plot is averaged over the two folds. 


