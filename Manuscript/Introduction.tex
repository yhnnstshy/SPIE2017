One in six men will develop prostate cancer (PCa) in their lifetime. An estimated 232,090 new cases and 30,350 deaths were expected for the year 2005, making PCa the second most common cause of cancer related deaths in men \cite{jemal2005cancer}. The state-of-the-art diagnostic method uses multiparameteric MR imaging (mpMRI) with transrectal ultrasound-guided biopsy \cite{costa2015mr}. MpMRI is the most accurate imaging method for PCa detection \cite{cornud2012value}; however, it requires the expertise of experienced radiologists, and as such, there is inconsistency across readers of varying experience \cite{ruprecht2012mri}. To increase inter-reader agreement and sensitivity, we developed a computer-aided detection (CAD) system that can automatically detect lesions on mpMRI that readers can use as a reference. Many of the existing prostate CAD systems use hand-crafted features to differentiate between cancerous and normal tissue \cite{carlsson2012prostate,kwak2015automated,litjens2014computer,wang2014computer}. However, due to the recent success in deep convolutional neural netowrks (DCNN) architectures that can learn descriptive features from the given data \cite{lecun2015deep}, we investigated a DCNN-based prostate-CAD ($CAD_{DL}$) to find an improved solution for PCa detection on mpMRI.

$CAD_{DL}$ had a 0.897 area-under-the-curve (AUC) for a response operating characteristic analysis after training for 6 epochs, which translated to a 0.845 detection rate at 0.2 false-positive rate or 0.94 detection rate at 10 false positives per patient on the free response operating characteristic (FROC) curve. A competing support vector machine based CAD ($CAD_{SVM}$) that uses hand-crafted features had an inferior performance with 0.859 AUC translating to a 0.801 detection at the same 0.2 false positive rate or 0.85 detection for 10 false positives per patient. Qualitative analysis also showed that $CAD_{DL}$ detects lesions with a higher confidence than $CAD_{SVM}$; albeit, it is at times prone to false positive detections arising from benign prostatic hyperplasia (BPH) presence and artifacts within the image. 

Our results suggest that, DCNN architectures have the potential to improve prostate cancer detection on mpMRI. $CAD_{DL}$ was able to learn discriminating features from the given data achieving competitive results. The free response operating characteristic analysis and generated prediction maps further elucidate its potential for clinical application. For future studies, we will attempt to develop a DCNN based prostate-CAD that has more applications in addition to tumor detection such as tumor size approximation and severity prediction.
