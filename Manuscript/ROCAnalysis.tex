ROC analysis is performed on 3D probability maps. The 3D probability maps are first formed for each case by taking the 2D probability map outputs from $CAD_{DL}$ and stacking them. Detection rate is then determined by computing the 90th percentile of the probability scores within each cancerous lesion volume. If this exceeds some threshold, then the cancerous lesion is said to be \textit{detected}. In other words, if at least 10\% of the lesion has relatively high probability, it is considered detected. False positive rate is then determined by placing a 3mm x 3mm x 3mm grid on the prostate. Only cells that are inside the prostate are considered. If the 90th percentile of probability scores in a cell exceeds a threshold, then the cell is said to be a false positive. The rationale for the grid is that a prostate reader will likely not mark an imperceptibly small region (e.g. a single voxel) to be biopsied. Furthermore, the grid cells that are within 3mm to the boundary of a cancerous lesion are ignored owing to possible ambiguity in the annotations. The final ROC curve is an average of the curves from the two folds.


