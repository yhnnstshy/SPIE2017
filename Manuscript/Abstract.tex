Prostate cancer (PCa) is the second most common cause of cancer related deaths in men. Multiparametric MRI (mpMRI) is the most accurate imaging method for PCa detection; however, it requires the expertise of experienced radiologists leading to inconsistency across readers of varying experience. To increase inter-reader agreement and sensitivity, we developed a computer-aided detection (CAD) system that can automatically detect lesions on mpMRI that readers can use as a reference. We investigated a convolutional neural network based deep-learing (DCNN) architecture to find an improved solution for PCa detection on mpMRI. We adopted a network architecture from a state-of-the-art edge detector that takes an image as an input and produces an image probability map. Two-fold cross validation along with a receiver operating characteristic (ROC) analysis and free-response ROC (FROC) were used to determine our deep-learning based prostate-CAD's ($CAD_{DL}$) performance. The efficacy was compared to an existing prostate CAD system that is based on hand-crafted features, which was evaluated on the same test-set. $CAD_{DL}$ had an 86\% detection rate at 20\% false-positive rate while the top-down learning CAD had 80\% detection rate at the same false-positive rate, which translated to 94\% and 85\% detection rate at 10 false-positives per patient on the FROC. A CNN based CAD is able to detect cancerous lesions on mpMRI of the prostate with results comparable to an existing prostate-CAD showing potential for further development.
